% Cover letter using letter.cls
\documentclass{letter}
\usepackage[colorlinks,urlcolor=blue,citecolor=blue,linkcolor=blue]{hyperref}
\usepackage{graphicx}
\usepackage[margin=0.5in, rmargin=1.in]{geometry}
\setlength{\textheight}{10in} % increase text height
\topmargin=-1.5in    % Make letterhead start about 1 inch from top of page
%\textheight=8in    % text height can be bigger for a longer letter
\oddsidemargin=0pt % leftmargin is 1 inch
\textwidth=6.5in   % textwidth of 6.5in leaves 1 inch for right margin
%\usepackage{helvetica} % uses helvetica postscript font (download helvetica.sty)
%\usepackage{newcent}   % uses new century schoolbook postscript font 
% the following commands control the margins:
\newcommand\mancenter[1]{\moveleft.5\hoffset\centerline{#1}}
\newcommand\boldline{\moveleft.0\hoffset\vbox{\hrule width\textwidth
    height 1pt}}
\newcommand\toedit[1]{\textcolor{red}{#1}} 
\begin{document}

 
\begin{letter}{Recruiting Manager\\
Quantlab Boston\\
10 Summer St.~Suite 602\\
Boston, MA, 02110\\
}


\mancenter{\Large \bf{Benjamin A. Cook}} \mancenter{ 60 Garden St.~MS
  10, Cambridge, MA 02138 $\bullet$ (603) 313-2888}
\mancenter{  \href{mailto:bcook@cfa.harvard.edu}{bcook@cfa.harvard.edu} $\bullet$
  \href{http://www.cfa.harvard.edu/~bcook}{www.cfa.harvard.edu/$\sim$bcook}}
\boldline

\signature{\vspace{-0.3in}Ben Cook}                  % name for signature 
\longindentation=0pt                     % needed to get closing flush left
\let\raggedleft\raggedright              % needed to get date flush left

\opening{Dear Recruiting Manager,} 
 
\noindent I am writing to express my interest in the Summer 2018
Quantitative Research Internship with Quantlab Boston. I recently
spoke about the opportunities available at Quantlab with Dr.~Manish Gupta
at Harvard's Finance Career Networking night on Thursday August
31st. I am a 4th-year PhD candidate in astrophysics at Harvard, expect
to complete my PhD in May 2019, and am eager for the chance to get an
insider's view of working in quantitative finance at Quantlab this
summer. I believe my extensive training as an independent researcher,
my mindset for solving challenging problems with statistical methods,
and my experience using and teaching GPU-acceleration for improving
existing Python projects will make me a great asset to Quantlab as a
Quantitative Research Intern.

\noindent My skills and my interest in quantitative analysis both
grew from my early passion for astrophysics. During my undergraduate
studies at Princeton, my intellectual curiosity was centered on
questions surrounding the properties of super-massive black holes and
the nature of dark matter. But what excited me daily and kept me up
into the early-morning hours were challenges in computation and
analysis. To this day I remember the feeling of elation I felt when I
located the last bug from my Junior Thesis code modeling the
time-series photometry of the asteroid Iris. It was nearly 4AM, one
morning in early January 2012, and when my monitor finally displayed a
sine-curve showing a rotation period of 7.14($\pm0.012$) hours, I
couldn't help but get up and dance around the empty computer lab. I
truly find nothing more thrilling than solving a challenging problem
through careful planning, clever statistical analysis, and hard work.

\noindent Throughout my PhD program, I have solidified my foundations
as an independent researcher as well as expanded my computational
training through both course study and research applications. I have
taken graduate courses through the Computational Science and
Engineering program for five straight semesters, including
\textit{Stochastic Methods for Data Analysis, Inference and
  Optimization}, and \textit{Advanced Machine Learning}. I taught
myself the tools of GPU-acceleration with CUDA C and applied it to my
PhD thesis code to achieve a speed-up of over 16x compared to the
original Python. I have since become a participant in the NVIDIA
Developer and Educator program, have given several presentations on
the principles of GPU-acceleration, and was accepted to lead a
hands-on tutorial course at the upcoming .Astronomy 9 conference in
Cape Town, South Africa in November 2017.

\noindent I would be very excited for the opportunity to apply my
analytical skills and independence as a researcher to the Quantitative
Research Internship at Quantlab. As a PhD candidate looking for
eventual employment in quantitative finance, I would like to prove
myself capable to apply my knowledge to problems in industry, and
believe Quantlab is the ideal place to do so because of Quantlab's
proven dedication to using the skills of research scientists to
revolutionize the financial industry. I have also submitted an
application through the Harvard Campus Interview Program on Crimson
Careers, and would greatly appreciate the opportunity to speak more in
person. I look forward to hearing back from you and learning more
about the Quantitative Research Internship. Please feel free to
contact me at (603) 313-2888 or bcook@cfa.harvard.edu.

Thank you for your time and consideration.

%Later, when applying for full-time positions, speak more about
%Leadership experience

\closing{Sincerely,}

%\encl{Resume}					% Enclosures

\end{letter}
\end{document}












