% LaTeX file for CV 
% This file uses the resume document class (res.cls)

\documentclass{res}
\usepackage[colorlinks,urlcolor=blue,citecolor=blue,linkcolor=blue]{hyperref}
\usepackage[ampersand]{easylist}
\usepackage{graphicx, bold-extra, enumitem, amssymb}
%\usepackage{helvetica} % uses helvetica postscript font (download helvetica.sty)
%\usepackage{newcent}   % uses new century schoolbook postscript font  
\topmargin=-0.5in % start text higher on the page
\setlength{\textheight}{10in} % increase text height
\newcommand\mancenter[1]{\moveleft.5\hoffset\centerline{#1}}
\newcommand\boldline{\moveleft\hoffset\vbox{\hrule width\resumewidth height 1pt}}
\newcommand\mysubsections{\ListProperties(Hide=1000, Progressive*=6ex,
  Space*=-0.12in, Space=-0.12in, Space2*=-0.15in)}

\begin{document}  
\mancenter{\huge \bf{Benjamin A. Cook}}
\mancenter{\huge Curriculum Vitae} 

\begin{resume}                        

\section{\textbf{Contact Information}}
\vspace{.1in}
\begin{tabular}{@{}p{2.5in} p{2.5in}}
Mailing Address: & Email: \href{mailto:bcook@cfa.harvard.edu}{bcook@cfa.harvard.edu}\\
60 Garden St. MS 10 & Homepage:
\href{http://www.cfa.harvard.edu/~bcook}{www.cfa.harvard.edu/$\sim$bcook}\\
Cambridge, MA 02138 & Twitter: \href{https://twitter.com/bacook17}{@bacook17}\\
\end{tabular}


\section{\textbf{Education}}
\vspace{.1in}
\begin{easylist} \mysubsections
  & \textbf{Harvard University}, Cambridge, MA \hfill 2014 -- Present \hspace{.5in}
  
  && Ph.D.~(\textit{In progress}), Astronomy and Astrophysics

  && M.A.~(2016), Astronomy and Astrophysics

  && Secondary Field, Computational Science and Engineering

  & \textbf{Princeton University}, Princeton, NH \hfill 2010 -- 2014 \hspace{.5in}

  && A.B.~(2014), Astrophysical Sciences -- Magna cum laude

  && Thesis: \textit{Keep Calm and Baryon: The Distribution of Baryons
    and Dark Matter in the Universe}

  && Advisor: Prof.~Neta Bahcall

%  &\textbf{Keene High School}, Keene, NH \hfill 2006 -- 2010 \hspace{.5in}
  
%  && Class Valedictorian

\end{easylist}

\section{\textbf{Awards and Honors}}
\vspace{.2in}
\begin{easylist} \mysubsections
  & Certificate of Teaching Excellence, Derek Bok Center for
  Teaching and Learning (2016)

  && Awarded for Fall 2015 teaching of Harvard Astro 200 
  
  & NSF Graduate Research Fellowship (2014 -- Present)

  & Elected to Sigma Xi Science Honor Society (2014)

  & AAS Chambliss Medal (2014)

  && Awarded for Jan.~2014 poster presentation at AAS 223.
\end{easylist}

\section{\textbf{Teaching Experience}}
\vspace{.2in}
\begin{easylist} \mysubsections
  & Teaching Fellow, Harvard Astro 200 (Fall 2015)

  & Teaching Fellow, Harvard Astro 16 (Spring 2014)

  & Instructor, \textit{The Marvelous Universe}, Princeton
  Wintersession Course (Jan.~2014)

  & Teaching Assistant, Princeton AST 204 (Spring 2013)

  & Teaching Assistant, Princeton AST 205 (Fall 2012)

\end{easylist}

\section{\textbf{Professional and Outreach Activities}}
\vspace{0.2in}
\begin{easylist} \mysubsections
  & Mentor, Banneker Institute Summer Program (Summer 2016)

  & Coordinator, Harvard Astronomy prospective student visits (2016)
  
  & Local Organizing Committee,
  \href{http://www.comscicon.com}{ComSciCon} National Workshop (2014
  -- Present)

  && Co-Chair (2015 -- Present)

  & Author/Peer-Editor, \href{http://www.astrobites.org}{Astrobites}
  astronomy blog (2014 --
  Present)

  & Volunteer math tutor, Princeton Prison Teaching Initiative (Spring 2014)


  %& Volunteer, Mars and Beyond exhibit, Boston Museum of Science
  %(Aug.~2013)

  %& Member, The Planetary Society (2014 -- Present)

  & Attendee, AAS Astronomy Ambassador Training Workshop
  (Jan.~2014)

  & Volunteer, Peyton Observatory Public Observing Nights (2012
  -- 2014)

  & Junior Member, American Astronomical Society (2012 -- Present)

  & Departmental Representative, Princeton Major Choices Advising
  (2012 -- 2014)

  & Chapter President, Society of Physics Students (2012 -- 2014)
  \end{easylist}

\section{\textbf{Presentations}}
\vspace{0.2in}
\begin{easylist} \mysubsections
  & \textbf{Contributed Talks}
  && 228th Meeting of the AAS, \#202.01 -- June 2016, San
  Diego, CA &&& \textit{The Information Content of Stellar Halos: Accretion
    Histories and Stellar Population Gradients in Quiescent Illustris
    Galaxies} 
  
  && International Astronomy Union Symposium
  317, \#2246021 -- August 2015, Honolulu, HI &&& \textit{Stellar Populations of Stellar Halos: Results from the
    Illustris Simulation}
 
 & \textbf{Outreach Talks}
  
  && CfA Summer Colloquium Series -- Summer
  2016, Cambridge, MA &&& \textit{TBD}
  
  && New Hampshire Astronomical Society Meeting -- Summer
  2016, Manchester, NH &&& \textit{TBD}
  
  & \textbf{Posters}
  
  && 3rd Annual GMT Community Science Meeting --
  October 2015, Pacific Grove, CA &&& \textit{Stellar Populations of Stellar Halos: Results from the
    Illustris Simulation}
  
  && International Astronomy Union Symposium
  317, \#S317p.12 -- August 2015, Honolulu, HI &&& \textit{Stellar Populations of Stellar Halos: Results from the
    Illustris Simulation}
  
  && 223rd Meeting of the AAS, \#441.10 -- January 2014, Washington,
  DC &&& \textit{Magnetic Dynamos and X-ray Activity in Ultracool Dwarfs:
    Constraining the Role of Rotation} -- \textbf{Chambliss Medal Winner}
  
  && The 4th Tri-State Astronomy Conference at CUNY -- September 2013,
  New York &&& \textit{Magnetic Dynamos and X-ray Activity in Ultracool Dwarfs:
    Constraining the Role of Rotation}
\end{easylist}

\section{\textbf{Publications}}
\vspace{.2in}
\begin{easylist}[enumerate] \mysubsections
  \ListProperties(Hide=0)
  & \textbf{Cook, B.A.}, Conroy, C., Pillepich, A., Hernquist,
  L. 2016, ``Stellar populations of stellar halos: Results from the
  Illustris simulation'', \textit{Proceedings of IAUS 317} [\href{http://arxiv.org/abs/1509.05036}{arXiv:1509.05036}]
  
  & \textbf{Cook, B.A.}, Williams, P.K.G., Berger, E. 2014, ``Trends
  in Ultracool Dwarf Magnetism.~II. The Inverse Correlation between
  X-ray Activity and Rotation as Evidence for a Bimodal Dynamo'',
  \textit{ApJ}, 785, 10
         [\href{http://arxiv.org/abs/1310.6758}{arXiv:1310.6758}]

  & Williams, P.K.G., \textbf{Cook, B.A.}, Berger, E. 2014, ``Trends
         in Ultracool Dwarf Magnetism.~I. X-ray Suppression and Radio
         Enhancement'', \textit{ApJ}, 785, 9
         [\href{http://arxiv.org/abs/1310.6757}{arXiv:1310.6757}]

  & P\^{a}ris, I., Petitjean, P., Aubourg, \'E., et al. 2014,
         ``The Sloan Digital Sky Survey quasar catalog: tenth data
         release'', \textit{A\&A}, 563, A54
         [\href{http://arxiv.org/abs/1311.4870}{arXiv:1311.4870}]
\end{easylist}
\NewList

\section{\textbf{Selected Research Experience}}
\vspace{0.2in}
\begin{easylist} \mysubsections
  & Research Exam Project (Harvard University) \hfill Fall 2015 --
  present \hspace{0.5in}

  && Topic: Galactic accretion histories and stellar populations in
  hydrodynamical simulations.

  && Advisor: Prof.~Charlie Conroy

  & Senior Thesis (Princeton University) \hfill  Fall 2013 -- Spring
  2014 \hspace{0.5in}

  && Topic: The cosmic distributions of baryons and dark matter.

  && Advisor: Prof.~Neta Bahcall

  & Astronomy REU (Harvard University) \hfill Summer 2013 \hspace{0.5in}

  && Topic: The X-ray activity/rotation relation in ultracool
  dwarfs

  && Advisors: Drs.~Edo Berger and Peter Williams

  & Junior Research Paper (Princeton University) \hfill Spring 2013 \hspace{0.5in}

  && Topic: Type II quasars in the BOSS survey

  && Advisor: Prof.~Michael Strauss

  %& Junior Research Paper (Princeton University) \hfill Fall
  %2012 \hspace{0.5in}

  %&& Topic: Photometric analysis of asteroids with the HATNet
  %survey

  %&& Advisor: Prof.~G\'asp\'ar Bakos

  %& Undergraduate Summer Research Program (Princeton University)
  %\hfill Summer 2012 \hspace{0.5in}

  %&& Topic: Galactic luminosity and mass functions from simulations

  %&& Advisor: Dr.~Renyue Cen
\end{easylist}

\section{\textbf{Computing Skills}}
\vspace{.2in}
\begin{easylist} \mysubsections
  & Languages

  && Python, C, Java, Wolfram (Mathematica)

  & Electronic Presentation
  
  && \LaTeX, HTML, Jupyter (iPython) notebook

  & Other Development Tools

  && Bash, Git, Make, Python multiprocessing, SLURM cluster manager
\end{easylist}

\end{resume} 
\end{document}

