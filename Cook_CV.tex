% LaTeX file for CV 
% This file uses the resume document class (res.cls)

\documentclass{res}
\usepackage[colorlinks,urlcolor=blue,citecolor=blue,linkcolor=blue]{hyperref}
\usepackage{graphicx, bold-extra}
%\usepackage{helvetica} % uses helvetica postscript font (download helvetica.sty)
%\usepackage{newcent}   % uses new century schoolbook postscript font  
\topmargin=-0.5in % start text higher on the page
\setlength{\textheight}{10in} % increase text height
\newcommand\mancenter[1]{\moveleft.5\hoffset\centerline{#1}}
\newcommand\boldline{\moveleft\hoffset\vbox{\hrule width\resumewidth height 1pt}}

\begin{document}  
\mancenter{\huge \bf{Benjamin A. Cook}}
\mancenter{\huge Curriculum Vitae} 

\begin{resume}                        

\section{\textbf{Contact Information}}
\vspace{.1in}
\begin{tabular}{@{}p{2.5in} p{2.5in}}
Mailing Address: & Email: \href{mailto:bcook@cfa.harvard.edu}{bcook@cfa.harvard.edu}\\
60 Garden St. MS 10 & Homepage:
\href{http://www.cfa.harvard.edu/~bcook}{www.cfa.harvard.edu/$\sim$bcook}\\
Cambridge, MA 02138 & Twitter: \href{https://twitter.com/bacook17}{@bacook17}\\
\end{tabular}


\section{\textbf{Education}}
\vspace{.1in}
\begin{tabular}{@{}p{4in} r@{}}
\textbf{Harvard University}, Cambridge, MA & 2014 -- Present
\end{tabular}
\begin{itemize} \itemsep -2pt
\item[] Ph.D.~(\textit{in progress}) Astronomy and Astrophysics
\end{itemize}

\begin{tabular}{@{}p{4in} r@{}}
\textbf{Princeton University}, Princeton, NJ & 2010 -- 2014
\end{tabular}
\begin{itemize} \itemsep -2pt
\item[] A.B.~Astrophysical Sciences -- Magna cum laude
\item[] Thesis: \textit{Keep Calm and Baryon: The Distribution of
  Baryons and Dark Matter in the Universe}
\item[] Advisor: Prof.~Neta Bahcall
\end{itemize}

\begin{tabular}{@{}p{4in} r@{}}
  Keene High School, Keene, NH & 2006 -- 2010
\end{tabular}
\begin{itemize} \itemsep -2pt
\item[] Class Valedictorian
\end{itemize}

\section{\textbf{Awards and Honors}}
\vspace{.1in}
Elected to Sigma Xi Science Honor Society (2014)\\
NSF Graduate Research Fellowship (2014 -- Present)\\
AAS Chambliss Medal (2014)


\section{\textbf{Teaching and Outreach}} 
\vspace{.1in}
Volunteer, Harvard Observing Program (Fall 2014 -- Present)\\
Volunteer math tutor, Prison Teaching Initiative (Spring 2014)\\
Instructor, \textit{The Marvelous Universe}, Princeton Wintersession Course (Jan.~2014)\\
Participant, AAS Astronomy Ambassador Training Workshop (Jan.~2014)\\
Volunteer, Mars and Beyond exhibit, Boston Museum of Science (Aug.~2013)\\
Teaching Assistant, Princeton AST 204 (Spring 2013)\\
Teaching Assistant, Princeton AST 205 (Fall 2012)\\
Volunteer, Peyton Observatory Public Observing Nights (2012
-- 2014)

\section{\textbf{Professional Activities}}
\vspace{0.1in}
Author, \href{http://www.astrobites.org}{Astrobites} (2014 -- Present)\\
Member, The Planetary Society (2014 -- Present)\\
Junior Member, American Astronomical Society (2012 -- Present)\\
Departmental Representative, Princeton Major Choices Advising (2012 -- 2014)\\
Chapter President, Society of Physics Students (2012 -- 2014)


\section{\textbf{Publications}}
\vspace{.1in} \textbf{Cook, B.A.}, Williams, P.K.G., Berger, E. 2014,
``Trends in Ultracool Dwarf Magnetism.~II. The Inverse Correlation
between X-ray Activity and Rotation as Evidence for a Bimodal
Dynamo'', \textit{ApJ}, 785, 10
[\href{http://arxiv.org/abs/1310.6758}{arXiv:1310.6758}]

Williams, P.K.G., \textbf{Cook, B.A.}, Berger, E. 2014, ``Trends in
Ultracool Dwarf Magnetism.~I. X-ray Suppression and Radio
Enhancement'', \textit{ApJ}, 785, 9
[\href{http://arxiv.org/abs/1310.6757}{arXiv:1310.6757}]

P\^{a}ris, I., Petitjean, P., Aubourg, \'E., et al. 2014, ``The Sloan
Digital Sky Survey quasar catalog: tenth data release'',
\textit{A\&A}, 563, A54
       [\href{http://arxiv.org/abs/1311.4870}{arXiv:1311.4870}]

\section{\textbf{Posters}}
\vspace{.1in}
\textit{Magnetic Dynamos and X-ray Activity in Ultracool Dwarfs (UCDs):
Constraining the Role of Rotation}\\ 223rd Meeting of the AAS,
\#441.10 -- January 2014, Washington, DC \\ \textbf{Chambliss Medal
  Winner}

\textit{Magnetic Dynamos and X-ray Activity in Ultracool Dwarfs
  (UCDs): Constraining the Role of Rotation}\\ The Fourth Tri-State
Astronomy Conference at CUNY -- September 2013, New York

\section{\textbf{Research Experience}}
\vspace{0.1in}
\begin{tabular}{@{}p{4in} r}
  Research Exam Project (Harvard University) & Fall 2015 -- present
\end{tabular}
\begin{itemize} \itemsep -2pt
  \item[] Topic: Galactic stellar halos and merger histories in
    numerical simulations.
  \item[] Advisor: Prof.~Charlie Conroy
\end{itemize}
\begin{tabular}{@{}p{4in} r}
  Senior Thesis (Princeton University) & Fall 2013 -- Spring 2014
\end{tabular}
\begin{itemize} \itemsep -2pt
\item[] Topic: The cosmic distributions of baryons and dark matter.
\item[] Advisor: Prof.~Neta Bahcall
\end{itemize}
\begin{tabular}{@{}p{4in} r}
  Astronomy REU (Harvard University) & Summer 2013
\end{tabular}
\begin{itemize} \itemsep -2pt
\item[] Topic: The X-ray activity/rotation relation in ultracool
  dwarfs
\item[] Advisors: Drs.~Edo Berger and Peter Williams
\end{itemize}
\begin{tabular}{@{}p{4in} r}
  Junior Research Paper (Princeton University) & Spring 2013
\end{tabular}
\begin{itemize} \itemsep -2pt
\item[] Topic: Type II quasars in the BOSS survey
\item[] Advisor: Prof.~Michael Strauss
\end{itemize}
\begin{tabular}{@{}p{4in} r}
  Junior Research Paper (Princeton University) & Fall 2012
\end{tabular}
\begin{itemize} \itemsep -2pt
\item[] Topic: Photometric analysis of asteroids with the HATNet
  survey
\item[] Advisor: Prof.~G\'asp\'ar Bakos
\end{itemize}
\begin{tabular}{@{}p{4in} r}
  Undergraduate Summer Research Program (Princeton University) &
  Summer 2012
\end{tabular}
\begin{itemize} \itemsep -2pt
\item[] Topic: Galactic luminosity and mass functions from simulations
\item[] Advisor: Dr.~Renyue Cen
\end{itemize}

\section{\textbf{Computing Skills}}
\vspace{.1in}
Programming Languages
\begin{itemize}
\item[] Python, C, Java
\end{itemize}
Electronic Presentation
\begin{itemize}
\item[] \LaTeX, HTML
\end{itemize}
Others
\begin{itemize}
\item[] Linux/UNIX, \textit{Chandra} CIAO, Mathematica, GIT, FITS
\end{itemize}

\end{resume} 
\end{document}

